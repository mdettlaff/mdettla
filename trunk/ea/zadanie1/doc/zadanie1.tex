\documentclass[11pt]{article}
\usepackage[utf8]{inputenc}
\begin{document}
\title{Algorytmy ewolucyjne. Zadanie 1}
\author{Michal Dettlaff}
\maketitle

\section{Wyniki dzialania algorytmu}

\noindent
Domyslne parametry algorytmu:\newline
Selekcja turniejowa o rozmiarze turnieju 4\newline
Mutacja 1-punktowa z prawdopodobienstwem 1/n\newline
Krzyzowanie 1-punktowe z prawdopodobienstwem 0.6\newline
Rozmiar populacji $ \mu/4 = 112 $\newline
\newline
Najlepsze rozwiazanie po 30 powtorzeniach:\newline
Wartosc przystosowania = X\newline
Srednie przystosowanie: X\newline
Odchylenie standardowe: X\newline
Czas wykonania: X s

\section{Pytania dodatkowe}

\begin{description}

\item[a)] Wyniki po zwiekszeniu rozmiaru populacji do $ \mu = \lfloor n/2 \rfloor $\newline
Wartosc przystosowania = X\newline
Srednie przystosowanie: X\newline
Odchylenie standardowe: X\newline
Czas wykonania: X s
\item[b1)] Wyniki po zastosowaniu krzyzowania 2-punktowego\newline
Wartosc przystosowania = X\newline
Srednie przystosowanie: X\newline
Odchylenie standardowe: X\newline
Czas wykonania: X s
\item[b2)] Wyniki po zrezygnowaniu z krzyzowania\newline
Wartosc przystosowania = X\newline
Srednie przystosowanie: X\newline
Odchylenie standardowe: X\newline
Czas wykonania: X s
\item[c)] Wyniki po zastosowaniu selekcji rangowej\newline
Wartosc przystosowania = X\newline
Srednie przystosowanie: X\newline
Odchylenie standardowe: X\newline
Czas wykonania: X s
\item[d)] Jak mozna usprawnic algorytm?\newline
Mozna inicjowac populacje poczatkowa osobnikami nie calkowicie losowymi, ale
majacymi rozmiar zbioru niezaleznego zblizony do oczekiwanego w rozwiazaniu.
\item[e)] Porownac skutecznosc algorytmu genetycznego z innym (znanym)
algorytmem wyznaczania maksymalnego zbioru niezaleznego.\newline
Algorytm genetyczny dziala wolniej.
\item[f)] Praktyczne zastosowania rozwiazywanego problemu.\newline
Znajdowanie maksymalnych zbiorow niezaleznych moze miec zastosowanie w wielu
algorytmach zwiazanych z problemami NP-trudnymi.\newline
Problem maksymalnego zbioru niezaleznego znajduje zastosowanie w wielu dziedzinach,
takich jak:\newline
Rozpoznawanie obrazow.\newline
Biologia molekularna.\newline
Zadanie harmonogramowania.\newline
Oznaczanie map.
\end{description}

\end{document}
