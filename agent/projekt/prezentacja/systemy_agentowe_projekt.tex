\documentclass{beamer}
\usepackage{polski}
\usepackage[utf8]{inputenc}
\usepackage[OT4]{fontenc}

\usepackage{beamerthemesplit}
\usetheme{Boadilla}

\title{Modelowanie aukcji groszowych za pomocą systemu Jadex}
\author{Michał Dettlaff, Bartłomiej Kossakowski}
\date{\today}

\begin{document}

\frame{\titlepage}

%\section[Outline]{}
%\frame{\tableofcontents}

%\section{Introduction}
%\subsection{Overview of the Beamer Class}

\frame {
  \frametitle{Jadex}
  Implementacja modelu BDI:
  \begin{itemize}
    \item Belief
    \item Desire / Goal
    \item Intention / Plan
  \end{itemize}
}

\frame {
  \frametitle{Aukcje groszowe}
  Przykłady:
  \begin{itemize}
    \item podbij.pl
    \item fruli.pl
    \item za10groszy.pl
    \item swoopo.com
    \item BidRivals.com
  \end{itemize}
}

\frame {
  \frametitle{Aukcje groszowe}
  Każda oferta (podbicie, bid):
  \begin{itemize}
    \item podnosi cenę przedmiotu o 1 grosz
    \item przedłuża czas aukcji o 20 sekund
    \item kosztuje kupującego 50 groszy
  \end{itemize}
  \textit{Wartości liczbowe na przykładzie podbij.pl}
}

\frame {
  \frametitle{Protokół aukcji}
  \begin{block}{Rejestracja na stronie aukcyjnej}
    \begin{tabular}{ll}
      Performative & subscribe \\
      Content      & \textit{nazwa\_agenta}
    \end{tabular}
  \end{block}
  \begin{block}{Odpowiedź na rejestrację}
    \begin{tabular}{ll}
      Performative & agree
    \end{tabular}
  \end{block}
  albo
  \begin{block}{Odpowiedź na rejestrację}
    \begin{tabular}{ll}
      Performative & disagree
    \end{tabular}
  \end{block}
}
\end{document}
